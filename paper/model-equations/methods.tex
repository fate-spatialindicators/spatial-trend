\documentclass[12pt]{article}
\usepackage[utf8]{inputenc}
\usepackage[margin=1.25in]{geometry}
\usepackage{amsmath, amsfonts, amssymb}
\usepackage{subfig}
\usepackage[dvipsnames]{xcolor}
\definecolor{niceblue}{HTML}{236899}
\usepackage{hyperref}
\hypersetup{
    colorlinks=true,
    linkcolor=black,
    filecolor=black,      
    urlcolor=niceblue,
    citecolor=niceblue,
    linkbordercolor = white
}
\usepackage{amsbsy}
\usepackage{epsfig}
\usepackage{subfig}
\usepackage{bm}
\usepackage{xspace}
\usepackage{color}
\usepackage{subfloat}
\usepackage{lineno}

\definecolor{darkgrey}{HTML}{A9A9A9}
\renewcommand\linenumberfont{\normalfont\bfseries\small\color{darkgrey}}

\usepackage{booktabs}
\usepackage[round]{natbib}
\bibliographystyle{fishfish}
\usepackage{geometry} % see geometry.pdf on how to lay out the page. There's lots.
\geometry{a4paper} % or letter or a5paper or ... etc
% \geometry{landscape} % rotated page geometry

% See the ``Article customise'' template for come common customisations

\title{}
\author{}
\date{} % delete this line to display the current date

\usepackage[round]{natbib}

\begin{document}

We modelled the response $y_{s,t}$ (catch per unit effort [CPUE] at point in space $s$ and time $t$) with a Tweedie distribution and a log link \citep{tweedie1984, dunn2005, anderson2019synopsis}
\begin{align}
  y_{s,t} &\sim \operatorname{Tweedie} \left(\mu_{s,t}, p, \phi \right), \: 1 < p < 2\ ,\\
  \mu_{s,t} &= \exp 
  \left(\alpha_t + \beta_1 D_{s,t} + \beta_2 D_{s,t}^2 + \omega_s + \epsilon_{s,t}  + \zeta_s t \right),\\
  \bm{\omega} &\sim \operatorname{MVNormal} \left( \bm{0}, \bm{\Sigma}_\omega \right),\\
  \bm{\epsilon}_t &\sim \operatorname{MVNormal} \left( \bm{0}, \bm{\Sigma}_{\epsilon} \right),\\
   \bm{\zeta} &\sim \operatorname{MVNormal} \left( \bm{0}, \bm{\Sigma}_\zeta \right),
\end{align}
where $\mu$ represents the mean, $p$ represents the power parameter, and $\phi$ represents the dispersion parameter.
The $\alpha_t$ parameters represent means for each year, and $\beta_{1}$ and $\beta_{2}$ represent coefficients for log depth ($D$) and log depth squared ($D^2$).
The symbols $\omega_s$ and $\epsilon_{s,t}$ represent spatial and spatiotemporal random effects drawn from Gaussian Markov random fields \citep{cressie2011} with covariance matrices $\bm{\Sigma}_{\epsilon}$ and $\bm{\Sigma}_\omega$.
The symbol $\zeta_s$ represents the spatially varying coefficients that represent local trends through time also drawn from Gaussian Markov random fields.
All three random fields have covariance matrices constrained by Mat\'ern covariance functions \citep{cressie2011} with independent scales but shared $\kappa$ parameters controlling the rate of decay of spatial correlation with distance \citep{cressie2011}.

Something about mesh \citep{rue2009, lindgren2011} \ldots 

something about optimization and REML \ldots

anisotropy \ldots

with the package sdmTMB \citep{anderson2019synopsis, sdmTMB}, which interfaces automatic differentiation in Template Model Builder \citep{kristensen2016} with INLA \citep{rue2009}

\bibliography{refs}

\end{document}